%%%%%%%%%%%%%%%%%%%%%%%%%%%%%%%%%%%%%%%%%%%%%%%%%%%%%%%%%%%%%%%%%%%%%%%%%%%%%%%%
%
%   based on Artikel-Vorlage (komplex)
%   https://stuga.math.uni-bremen.de/wiki/LaTeX:_Vorlagen
%
%%%%%%%%%%%%%%%%%%%%%%%%%%%%%%%%%%%%%%%%%%%%%%%%%%%%%%%%%%%%%%%%%%%%%%%%%%%%%%%%

\usepackage[ngerman,british]{babel}

\usepackage{amsmath,amssymb,amsthm,amsfonts,amsbsy,latexsym}
\usepackage{mathtools}
\usepackage{extarrows}
\usepackage[utf8]{inputenc}
\usepackage[T1]{fontenc}
\usepackage{enumerate,url}
\usepackage{graphicx}

\usepackage[sfdefault,lining]{FiraSans}
% \usepackage[fakebold]{firamath-otf}
% \renewcommand*\oldstylenums[1]{{\firaoldstyle #1}}

% \usepackage[sfdefault,scaled=.85]{FiraSans}
% \usepackage{newtxsf}

\usepackage{tikz}
\usepackage{pgfplots}

\usetikzlibrary{%
  automata,%
  positioning,%
}

\tikzset{
  every state/.style={
    draw=maincolor,
    thick,
    fill=maincolor!18,
    minimum size=0pt
  }
}

% \usepackage{bibgerm}
\usepackage[german=guillemets]{csquotes}
% \usepackage{makecell}
\usepackage{subcaption}
% \usepackage{eurosym}
\usepackage{lipsum}

% \usepackage[style=alphabetic,maxbibnames=99,maxalphanames=5]{biblatex}
% or
% \bibliographystyle{alphadin}

% \usepackage{selinput}
% \SelectInputMappings{
%   adieresis={ä},
%   germandbls={ß},
%   Euro={€},
% }

% Check if the abstract environment is already defined
\makeatletter
\@ifundefined{abstract}{
    % Redefine the abstract environment
    \newenvironment{abstract}{%
        \small
        \begin{center}
          {\normalfont\sectfont\nobreak\abstractname}
        \end{center}
        \quotation
    }{%
        \endquotation
    }
}{
    \PackageWarning{abstract}{Environment 'abstract' already defined.}
}
\makeatother


%%% MATH %%%%%%%%%%%%%%%%%%%%%%%%%%%%%%%%%%%%%%%%%%%%%%%%%%%%%%%%%%%%%%%%%%%%%%%
\newcommand{\fline}{\bigskip\noindent}
\newcommand{\field}[1]{\mathbb{#1}}
\newcommand{\N}{\field{N}}
\newcommand{\Z}{\field{Z}}
\newcommand{\Q}{\field{Q}}
\newcommand{\R}{\field{R}}
\newcommand{\C}{\field{C}}
\newcommand{\K}{\field{K}}
\newcommand{\B}{\field{B}}

\newtheorem*{fakt}{Fakt}
\newtheorem{fazit}{Fazit}
\newtheorem{satz}{Satz}
\newtheorem{theo}{Theorem}
\newtheorem{lemma}[satz]{Lemma}
\newtheorem{chara}{Charakterisierung}

\theoremstyle{definition}
\newtheorem{defi}{Definition}
\newtheorem{termi}{Terminologie}

\theoremstyle{remark}
\newtheorem*{beh}{Behauptung}
\newtheorem{rem}{Bemerkung}
\newtheorem*{folg}{Folgerung}
\newtheorem{korr}{Korollar}
\newtheorem*{bsp}{Beispiel}
\newtheorem*{bspe}{Beispiele}
\newtheorem{aufg}{\textbf{Aufgabenstellung}}

\newcommand{\induction}[1]{\textbf{Beweis} (durch vollständige Induktion nach $#1$): }
\newcommand{\proofcon}{\textbf{Beweis} (durch Kontraposition):}
\newcommand{\proofind}{\textbf{Beweis} (indirekt): }
\newcommand{\indb}[2]{Induktionsbeginn ($#1 = #2$): }
\newcommand{\indv}{Induktionsvoraussetzung ($\star$): }
\newcommand{\inds}[1]{Induktionsschritt ($#1\rightsquigarrow #1 +1$): }

\newcommand{\expo}[1]{\exp\left\{{#1}\right\}}
\newcommand{\series}[1]{\sum_{#1}^{\infty}}
\newcommand{\stetig}[2]{\mathscr{C}(#1,#2)}
\newcommand{\topologie}{\mathfrak{T}}

%%% PHYSICS %%%%%%%%%%%%%%%%%%%%%%%%%%%%%%%%%%%%%%%%%%%%%%%%%%%%%%%%%%%%%%%%%%%%
\usepackage{units}

%%% CS %%%%%%%%%%%%%%%%%%%%%%%%%%%%%%%%%%%%%%%%%%%%%%%%%%%%%%%%%%%
\usepackage{dirtree}

% \usepackage{listings}
% \usepackage{xcolor}

% \definecolor{codegreen}{rgb}{0,0.6,0}
% \definecolor{codegray}{rgb}{0.5,0.5,0.5}
% \definecolor{codepurple}{rgb}{0.58,0,0.82}
% \definecolor{backcolour}{rgb}{0.95,0.95,0.92}

% \lstdefinestyle{mystyle}{
%     backgroundcolor=\color{backcolour},
%     commentstyle=\color{codegreen},
%     keywordstyle=\color{magenta},
%     numberstyle=\tiny\color{codegray},
%     stringstyle=\color{codepurple},
%     basicstyle=\ttfamily\footnotesize,
%     breakatwhitespace=false,
%     breaklines=true,
%     captionpos=b,
%     keepspaces=true,
%     numbers=left,
%     numbersep=5pt,
%     showspaces=false,
%     showstringspaces=false,
%     showtabs=false,
%     tabsize=2
% }

% \lstset{style=mystyle}
